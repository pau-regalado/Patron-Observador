\documentclass{article}
\usepackage[utf8]{inputenc}
\usepackage{url}
\usepackage{hyperref}

\title{Informe sobre la Implementación del Patrón Observador con Weatherstack API}
\author{Tu Nombre}
\date{\today}

\begin{document}

\maketitle

\section{Introducción}
Este informe describe la implementación del patrón observador en una aplicación Java que utiliza la API de Weatherstack para obtener información meteorológica en tiempo real.

\section{Código Fuente}
Adjunto se encuentra el código fuente de la aplicación Java.

\subsection{WeatherstackExample.java}
\begin{verbatim}
// Aquí iría el código completo de WeatherstackExample.java
\end{verbatim}

\subsection{Observador.java}
\begin{verbatim}
// Aquí iría el código completo de la interfaz Observador
\end{verbatim}

\subsection{SujetoWeatherstack.java}
\begin{verbatim}
// Aquí iría el código completo de la clase SujetoWeatherstack
\end{verbatim}

\subsection{PantallaMeteorologica.java}
\begin{verbatim}
// Aquí iría el código completo de la clase PantallaMeteorologica
\end{verbatim}

\section{Uso de la Aplicación}
Para utilizar la aplicación, sigue los siguientes pasos:

\begin{enumerate}
    \item Compila y ejecuta el programa Java.
    \item Observa las actualizaciones en las pantallas meteorológicas cada vez que cambian las condiciones meteorológicas.
\end{enumerate}

\section{Conclusiones}
En este informe, se ha implementado el patrón observador para recibir actualizaciones en tiempo real de las condiciones meteorológicas a través de la API de Weatherstack. Este enfoque permite una arquitectura flexible y extensible para manejar eventos meteorológicos en una aplicación Java.

\end{document}
